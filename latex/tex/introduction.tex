% !TEX root = ../main.tex

This Advanced Physics Laboratory experience will guide you through the main steps that a physicist working in a large particle physics collaboration (like ATLAS, CMS, but also DUNE, NOVA, etc.) has to to do look at the experiment data and draw conclusions about the underlying physics. Specifically, you will be using simulated and real ATLAS proton-proton collisions at $\sqrt{s} = 13\ \TeV$ to do a series of particle physics measurements. Given the structure of the course you are following with us, your particle physics knowledge at the time of this experience will be limited: the first part of this document is a very quick and dirty introduction to the particle physics you need to know to be able to understand what you are doing when running the experiments. You will also need a review (or introduction, depending on previous experience) to some software tools that will be useful for this experience, but also for your future career, so long as it involves any software development and data analysis. 