% !TEX root = ../main_orange.tex

This Advanced Physics Laboratory experience will guide you through the main steps that a physicist working in a large particle physics collaboration (like ATLAS, CMS, but also DUNE, NOVA, etc.) has to to do look at the experiment data and draw conclusions about the underlying physics. The approach to data analysis is by the way not specific to particle physics: it is here presented in a particle physics context, but large collaborations in astrophysics and cosmology would use very similar techniques.

 Specifically, you will be using simulated and real ATLAS proton-proton collisions at $\sqrt{s} = 13\ \TeV$ to do a series of particle physics measurements. Given the structure of the course you are following with us, your particle physics knowledge at the time of this experience will be limited: the first part of this document is a very quick and dirty introduction to the particle physics you need to know to be able to understand what you are doing when running the experiments. You will also need a review (or introduction, depending on previous experience) of some software tools that will be useful for this experience, but also for your future career, so long as it involves any software development and data analysis. 

This handbook is organised as follows: this chapter lists the objectives of this experience. Chapter~\ref{sec:particle_physics} is a review of some of the particle physics you need to know. By no means this can be a substitute of a real particle physics course, but it should be enough to run these experiments. Section~\ref{sec:ATLAS} describes a bit more in detail a few specific aspects of particle detection. It also introduces you to some analysis techniques of a modern collider physics experiment. Section~\ref{sec:open_data} describes the practical aspects to set you up to run the experiments. It also discusses how the code should be written to take the most out of this experience. Finally, Section~\ref{sec:experiments} lists three experiments that should be fun to do and suitable for your level. 

Feedback on the handbook and on the specific experience is more than welcome, please send it to \verb|i.vivarelli@sussex.ac.uk|. 