% !TEX root = ../main_orange.tex

The main aim of the experience connected with this handbook is to get a glimpse of frontier particle physics. You will be \textit{seeing} the basic bricks with which our universe is composed. You will be observing $Higgs$ bosons, and top quarks and $W$ and $Z$ bosons. These particles form the foundations of our understanding of the electroweak interaction, mass, and stability of our universe. You will be asked to verify some of the properties of these particles, and hopefully by the end of the experience you will have a better understanding of the work that collider physicists do. 

\subsection{Learning Outcomes}

By the end of this experience: 

\begin{itemize}
\item You will have developed a good understanding of the experimental techniques used at a modern particle physics experiment through your preparation work.
\item You will have performed a cutting-edge physics experiments, and applied advanced analysis techniques when running the practical execution of the experiments.
\item You will have assessed the risks associated with a hypothetical running of ATLAS from its control room. 
\end{itemize}

\subsection{What you should deliver}

It is expected that the preparation and findings of your experience will be summarised in a logbook. The final logbook should be delivered in pdf. It does not matter whether you prepare it in word, latex, or any other text editor, but the final result should be a pdf file. Investing in learning (or improving) your latex skills is probably a good idea in view of your final year report: if you are a newbie, why don't you have a look at \href{https://www.overleaf.com}{Overleaf}? 

The logbook must contain the following entries: 

\begin{itemize}
\item{\textbf{Preparation:}} Before even starting to write a single line of code, you should dedicate an extensive amount of time to prepare. The preparation includes a number of different steps:
\begin{enumerate}
\item To improve your understanding of particle physics, you should read carefully Chapter~\ref{sec:particle_physics} and answer all exercises in Chapters~\ref{sec:particle_physics}, \ref{sec:ATLAS} and \ref{sec:open_data}. For each exercise, your logbook should contain details about your reasoning. If something is not clear, you should check the references provided. If things are still not clear, you should discuss with your tutor. 
\item You should expand on the contents of Chapter~\ref{sec:ATLAS}, and give a better description of the ATLAS detector. Chapter 1 of Ref.~\cite{Aad:2008zzm} can help a lot. 
\item You should have a risk assessment about a hypothetical data taking shift that you would take from the ATLAS Control Room. You can take some inspiration from the blog available at \href{https://atlas.cern/updates/atlas-blog/how-run-particle-detector}{this link}. Remember that ATLAS is buried 100 m into the ground below the control room. 
\end{enumerate}
\item{\textbf{Experiments:}} You should run the experiments in Chapter~\ref{sec:experiments}, document what you have done, and answer all questions as quantitatively as possible. The three experiments are in order of increasing difficulty. It is better to deliver the first two experiments well done and neglect the third one, rather than delivering three rushed experiments.
\end{itemize} 

